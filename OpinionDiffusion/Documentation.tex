\documentclass[]{article}
\usepackage{lmodern}
\usepackage{amssymb,amsmath}
\usepackage{ifxetex,ifluatex}
\usepackage{fixltx2e} % provides \textsubscript
\ifnum 0\ifxetex 1\fi\ifluatex 1\fi=0 % if pdftex
  \usepackage[T1]{fontenc}
  \usepackage[utf8]{inputenc}
\else % if luatex or xelatex
  \ifxetex
    \usepackage{mathspec}
  \else
    \usepackage{fontspec}
  \fi
  \defaultfontfeatures{Ligatures=TeX,Scale=MatchLowercase}
\fi
% use upquote if available, for straight quotes in verbatim environments
\IfFileExists{upquote.sty}{\usepackage{upquote}}{}
% use microtype if available
\IfFileExists{microtype.sty}{%
\usepackage{microtype}
\UseMicrotypeSet[protrusion]{basicmath} % disable protrusion for tt fonts
}{}
\usepackage[margin=1in]{geometry}
\usepackage{hyperref}
\hypersetup{unicode=true,
            pdftitle={Documentation},
            pdfauthor={Aadhithya Ramesh},
            pdfborder={0 0 0},
            breaklinks=true}
\urlstyle{same}  % don't use monospace font for urls
\usepackage{graphicx,grffile}
\makeatletter
\def\maxwidth{\ifdim\Gin@nat@width>\linewidth\linewidth\else\Gin@nat@width\fi}
\def\maxheight{\ifdim\Gin@nat@height>\textheight\textheight\else\Gin@nat@height\fi}
\makeatother
% Scale images if necessary, so that they will not overflow the page
% margins by default, and it is still possible to overwrite the defaults
% using explicit options in \includegraphics[width, height, ...]{}
\setkeys{Gin}{width=\maxwidth,height=\maxheight,keepaspectratio}
\IfFileExists{parskip.sty}{%
\usepackage{parskip}
}{% else
\setlength{\parindent}{0pt}
\setlength{\parskip}{6pt plus 2pt minus 1pt}
}
\setlength{\emergencystretch}{3em}  % prevent overfull lines
\providecommand{\tightlist}{%
  \setlength{\itemsep}{0pt}\setlength{\parskip}{0pt}}
\setcounter{secnumdepth}{0}
% Redefines (sub)paragraphs to behave more like sections
\ifx\paragraph\undefined\else
\let\oldparagraph\paragraph
\renewcommand{\paragraph}[1]{\oldparagraph{#1}\mbox{}}
\fi
\ifx\subparagraph\undefined\else
\let\oldsubparagraph\subparagraph
\renewcommand{\subparagraph}[1]{\oldsubparagraph{#1}\mbox{}}
\fi

%%% Use protect on footnotes to avoid problems with footnotes in titles
\let\rmarkdownfootnote\footnote%
\def\footnote{\protect\rmarkdownfootnote}

%%% Change title format to be more compact
\usepackage{titling}

% Create subtitle command for use in maketitle
\newcommand{\subtitle}[1]{
  \posttitle{
    \begin{center}\large#1\end{center}
    }
}

\setlength{\droptitle}{-2em}

  \title{Documentation}
    \pretitle{\vspace{\droptitle}\centering\huge}
  \posttitle{\par}
    \author{Aadhithya Ramesh}
    \preauthor{\centering\large\emph}
  \postauthor{\par}
      \predate{\centering\large\emph}
  \postdate{\par}
    \date{28 November 2018}


\begin{document}
\maketitle

\subsection{The Network}\label{the-network}

The network is comprised of users as nodes and retweets as the edges
between any two users.

\subsection{Dominance}\label{dominance}

Dominance is a metric used to measure the attention payed to a user or
the influence generated by a user.

Dominance is calculated by using the degree of each user and their
naighbours. Refer to the dominance function in dominance.R

\subsection{Creating the Network}\label{creating-the-network}

The network is created by the create\_network function present in the
parse\_edge\_datatype.R file.

It requires the tweets to be processed as the input.

It creates the edges and nodes dataframes and runs the dominance
function from the dominance.R file. The dominance is appended as a
vertex\_attribute in the nodes dataframe.

Both the dataframes are written in CSV format in the Data folder.

Return None.

\subsection{Dominance analysis}\label{dominance-analysis}

The dominance\_analysis function in dominance\_analysis.R does multiple
tasks pertaining to the analysis of the dominance distribution.

The first task is to sort the nodes by dominance.

The second task is to filter the dominance by threshold and store the
values.

Dominance distribution is represented by way of a histogram.

The next task is to create a dataframe which will include nodes possibly
belonging to a near clique by way of filtering using a degree derived
metric.

The last functionality provides a visualisation of the network with
dominance attributing to the size of the nodes.

Returns dominance\_histogram

\section{Timestamp network analysis}\label{timestamp-network-analysis}

The create\_timestamp\_network function in the time\_stamp\_network file
creates the network in the same manner as the create\_network function.
The function sorts the tweets by Created time before creation.

The dominance calculation is run at every interval (24 hours) and stored
as CSV.

\section{Cluster analysis}\label{cluster-analysis}

The cluster\_wise\_user\_analysis function in the
cluster\_wise\_user\_analysis file does the task of linking clusters to
individual users. It then links the attributes of the users to the same
table.


\end{document}
